%!TEX root = case-description.tex

\section{Evaluation}
\label{sec:EvaluationCriteria}

The evaluation will be done in two parts.
First, the submitted solutions will be evaluated by the case study authors on their correctness (the percentage of successful transformations) and performance (how long each transformation takes).
Second, all the TTC'15 participants will judge the solutions on their conciseness, readability and the overall usability and suitability of the transformation tool for the given task.

The solution is expected to be packed in the way that it is runnable from a command line shell.
The program should take two arguments, a path to a \emph{valid} Java 5 source file input and a path to where the result transformation should be stored.
%
\begin{minted}{sh}
$ ./my-solution.sh URLDownload.java SynthesizedURLDownload.java
\end{minted}

The main part of the evaluation is the assessment of the usability of a given transformation tool to the problem being addressed.
Usability of a programming language, a library or a tool is difficult to assess as it tends to be subjective since it largely depends on the preferences and background of its users.
We therefore leave it for the TTC attendees as they shall represent a mixture of preferences and backgrounds.
