%!TEX root = case-description.tex

\section{Case Description}
\label{sec:Description}

As the annotation library, we use a simplified version of \texttt{jcabi-aspects}\footnote{\url{http://aspects.jcabi.com/index.html}} which is a collection Java annotations together with AOP-based processing tool which allow developers to conveniently apply some common cross-cutting concerns into Java application.
From all the \texttt{jcabi-aspects} annotations, we have selected the following:
%
\begin{description}
  \item[\href{https://github.com/fikovnik/ttc15-tranj-case/blob/master/src/ttc15-tranj/src/main/java/ttc15/tranj/annotation/RetryOnFailure.java}{\javainline|@RetryOnFailure|}] for an automated retry of failed method execution.
  \item[\href{https://github.com/fikovnik/ttc15-tranj-case/blob/master/src/ttc15-tranj/src/main/java/ttc15/tranj/annotation/Cacheable.java}{\javainline|@Cacheable|}] for a simple data caching of parameterless methods return values.
  \item[\href{https://github.com/fikovnik/ttc15-tranj-case/blob/master/src/ttc15-tranj/src/main/java/ttc15/tranj/annotation/Loggable.java}{\javainline|@Loggable|}] for an automated logging of method calls including method entry, exit and any exception that is caught in the method body.
\end{description}

The core transformation involves traversing a source Java class and extending the annotated methods with the behavior specified by the above annotations.
The details of each annotation including the specification of its arguments are provided in its associated javadoc\footnote{\url{https://github.com/fikovnik/ttc15-tranj-case/tree/master/src/ttc15-tranj/src/main/java/ttc15/tranj/annotation}}.

In the following we provide an overview of the transformation tasks processing one annotation at a time.
The transformation is described using an example rather than with a formal definition.
While we believe that it should be enough, shall there be any ambiguity, we invite the reader to raise an issue on the case study \href{https://github.com/fikovnik/ttc15-tranj-case}{github page}.
In the last task, we look on the case of combining these annotations together and optimizing the resulting Java source code.

The complete code for these transformation tasks is provide on the github page in the \href{https://github.com/fikovnik/ttc15-tranj-case/tree/master/src/ttc15-tranj/src/main/java/ttc15/tranj/examples}{\javainline|ttc15.tranj.examples| package}.

\subsection{Running Example}

To better illustrate the transformation tasks of this case study, we use the following example.
In order to keep the case study concise and clear, we use only a single example and omit the use of any external library.

Let us consider a class that can download a content of an URL.
A possible Java implementation is shown in Listing~\ref{lst:URLDownload}.

\begin{listing}[H]
  \begin{javacode}
  public class URLDownload {
    private final URL url;

    public URLDownload(String url) throws MalformedURLException {
      this.url = new URL(url);
    }

    public byte[] get() throws IOException {
      try (InputStream input = url.openStream()) {

        ByteArrayOutputStream buffer = new ByteArrayOutputStream();
        byte[] chunk = new byte[4*1024];
        int n;

        while ((n = input.read(chunk)) > 0 ) {
          buffer.write(chunk, 0, n);
        }

        return buffer.toByteArray();
      }
    }
  }
  \end{javacode}
  \caption{Basic version of \javainline|URLDownload| class.}
  \label{lst:URLDownload}
\end{listing}

Now, when we have the basic functionality, we would like to extend it with following features:

\begin{itemize}[--]
   \item \emph{Failure handling.} The method should become more robust and accommodate for some of the inevitable network delays by retrying the download in case an exception occurs.
   It should however only retry the call in the case when it makes sense--\Ie when it is possible to recover.
   For example, in the case of \javainline|UnknownHostException| exception, it should fail immediately.
   \item \emph{Caching.} The URL should not be consulted every single time the method is called, but instead it should keep the content for certain amount of time in memory.
   \item \emph{Logging.} Each call to the method should be logged.
   The logging message should present the state of the current instance (\javainline|this.toString()|) as well as how long the invocation took.
\end{itemize} 

Instead of coding this manually, producing code such as the one shown in the listing in Section~\ref{sec:ManualURLDownload}, we would like to use annotations to declaratively specify the above concerns and have the a transformation tool automatically synthesize code similar to the shown in Section~\ref{sec:SynthesizedURLDownload}.
Concretely, the only change to the original code should the annotating the method with the following annotations:
%
\begin{javacode}
@RetryOnFailure(attempts = 3, delay = 1000, escalate = { UnknownHostException.class })
@Cacheable(lifetime = 1000)
@Loggable
public byte[] get() throws IOException { ... }
\end{javacode}

\subsection{Task 1: Retrying on a Failure}

The first transformation tasks should handle the \javainline|@RetryOnFailure| annotation.
The objective is extend a given method with a simple failure handling strategy that retries failed method invocations.
Annotating the \javainline|URLDownload.get()| method with
%
\begin{javacode}
@RetryOnFailure(attempts = 3, delay = 1000, types = { IOException.class }, escalate = { UnknownHostException.class })
\end{javacode}
%
should produce the following code:
%
\begin{javacode}
public byte[] get() throws IOException { 
  int __retryCount = 0;

  while (true) {
    try {
      // the content of the original method
    } catch (UnknownHostException e) {
      throw e;
    } catch (IOException e) {
      __retryCount += 1;

      if (__retryCount > 3) {
        throw e;
      } else {
        try {
          Thread.sleep(1000);
        } catch (InterruptedException e1) {
          throw e;
        }
      }
    }
  }
}
\end{javacode}

The original method code--\Ie what is between the lines 9--20 in Listing~\ref{lst:URLDownload}, is wrapped by another try-catch block which is itself in a while loop.
The number of attempts (3) is projected in the condition \javainline|if (__retryCount > 3)|, the delay is translated in a current thread sleep on lines 15--19.
Finally, we make a distinction in the exceptions that on which the call should be retried (\javainline|IOException|) and the ones that should be escalated (\javainline|UnknownHostException|) in the generated catch clauses in line 9 and 7 respectively.

\subsection{Task 2: Caching}

The second transformation tasks should handle the \javainline|@RetryOnFailure| annotation.
The objective is extend a given method with a simple failure handling strategy that retries failed method invocations.
Annotating the \javainline|URLDownload.get()| method with
%
\begin{javacode}
@Cacheable(lifetime = 1000)
\end{javacode}
%
should produce the following code:
%
\begin{javacode}
public class URLDownload {
  private long __getCacheLastAccessed = 0;
  private byte[] __getCacheContent = null;

  ...

  public byte[] get() throws IOException {
    if (System.currentTimeMillis() - __getCacheLastAccessed < 1000 && __getCacheContent != null) {
      return __getCacheContent;
    }

    try (InputStream input = url.openStream()) {
      ByteArrayOutputStream buffer = new ByteArrayOutputStream();
      byte[] chunk = new byte[4 * 1024];
      int n;

      while ((n = input.read(chunk)) > 0) {
        buffer.write(chunk, 0, n);
      }

      __getCacheContent = buffer.toByteArray();
      __getCacheLastAccessed = System.currentTimeMillis();
      
      return __getCacheContent;
    }
  }
}
\end{javacode}

The \javainline|__getCacheLastAccessed| and \javainline|__getCacheContent| defined on line 2 and 3 are cache bookkeeping fields used for storing the information about the last access time and method result respectively.
The condition on line 8 checks the validity of the cache based on the annotation \javainline|lifetime|.
Finally, the method exit point (\javainline|return|) is replaced with the stored result on lines 22--25.
This should be done for all method exit points.

\subsection{Task 3: Logging}

The last annotation transformation adds logging.
Any method annotated with \javainline|@Loggable| should have conditionally (depending on the annotation options) logged
\begin{inparaenum}[(i)]
\item its entry point, 
\item all its exit points, and 
\item all exceptions. 
\end{inparaenum}

Annotating the \javainline|URLDownload.get()| method with
%
\begin{javacode}
@Loggable
\end{javacode}
%
should produce the following code:
%
\begin{javacode}
public class URLDownload {
  // added logger
  private final org.slf4j.Logger __logger = org.slf4j.LoggerFactory.getLogger(URLDownload.class);

  ...

  public byte[] get() throws IOException {
    // added entry point logging
    long __entryTime = System.currentTimeMillis();
    if (__logger.isTraceEnabled()) {
      __logger.trace(String.format("get() [url='%s']: entry", url));
    }

    try (InputStream input = url.openStream()) {

      ByteArrayOutputStream buffer = new ByteArrayOutputStream();
      byte[] chunk = new byte[4 * 1024];
      int n;

      while ((n = input.read(chunk)) > 0) {
        buffer.write(chunk, 0, n);
      }

      // added exit point logging
      if (__logger.isTraceEnabled()) {
        __logger.trace(String.format("get(): exit in %d ms", System.currentTimeMillis() - __entryTime));
      }
      return buffer.toByteArray();
    }
  }
}
\end{javacode}

Line 2 defines an instance of SLF4J logger\footnote{\url{http://www.slf4j.org/}}.
Lines 7--10 and 22-24 insert the entry and exit point logging.

Because, there is no exception handling code--\Ie no catch clause, only entry and exit points are traced.
If we would however apply the logging annotation on the method shown as output of \javainline|@RetryOnFailure|, the result should look like:
%
\begin{javacode}
while (true) {
  try {
    ...
  } catch (UnknownHostException e) {
    // added exception handling
    __logger.error("get(): exception", e);

    throw e;
  } catch (IOException e) {
    // added exception handling
    __logger.error("get(): exception", e);

    __retryCount += 1;

    if (__retryCount > 3) {
      throw e;
    } else {
      try {
        Thread.sleep(1000);
      } catch (InterruptedException e1) {
        // added exception handling
        __logger.error("get(): exception", e);

        throw e;
      }
    }
  }
}
\end{javacode}

\subsection{Task 4: Annotation composition}

It should be possible to have all the extension present together on one method:
%
\begin{javacode}
@RetryOnFailure(attempts = 3, delay = 1000, escalate = { UnknownHostException.class })
@Cacheable(lifetime = 1000)
@Loggable
public byte[] get() throws IOException { ... }
\end{javacode}

The code transformation should be performed in order--\Ie, first applying \javainline|@RetryOnFailure|, then \javainline|@Cacheable| and finally adding \javainline|@Loggable|.
The expected result is shown in Section~\ref{sec:SynthesizedURLDownload}.

\subsection{Extension 1: Optimization (\emph{optional})}

A possible extension to this case study is to have the annotations being aware of one another.
In this case, the order of the annotations does not matter and they will be always applied in the order specified in the task 4.
For the caching and retry on failure, nothing will change, but the logging will recognize the generated code and adjust the logging messages appropriately.
The expected result--\Ie placement and message content is shown in the Section~\ref{sec:ManualURLDownload}.